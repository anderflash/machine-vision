\documentclass[9pt, twocolumn]{extarticle}
\usepackage{lipsum}
\usepackage[top=1cm, bottom=1.75cm, left=0.8cm, right=0.8cm]{geometry}
\usepackage{titlesec}
\usepackage{titling}
\usepackage{fourier}
\usepackage{multicol}
\usepackage{dsfont}
\usepackage{xcolor}
\usepackage{enumitem}   

\usepackage{amsmath, amsthm}

\setlist[enumerate]{nosep,noitemsep, topsep=0pt}

\title{Time of Flight Cameras}
\author{Anderson Tavares, anderson.moreira.tavares@liu.se}

\theoremstyle{definition}
\newtheorem{definition}{Definition}

%\setlength{\parskip}{0pt}
%\setlength{\parsep}{0pt}
%\setlength{\headsep}{0pt}
%\setlength{\topskip}{0pt}
%\setlength{\topmargin}{0pt}
\setlength{\topsep}{0pt}
%\setlength{\partopsep}{0pt}
\setlength{\droptitle}{-12em}   % This is your set screw
\titlespacing*{\section}
{0pt}{1.5ex plus 1ex minus .2ex}{0.0ex plus .0ex}
\titlespacing*{\subsection}
{0pt}{5.5ex plus 1ex minus .2ex}{0.0ex plus .0ex}
\renewcommand{\baselinestretch}{0.8}
\newcommand{\inlineeqnum}{\refstepcounter{equation}~~\mbox{(\theequation)}}

  
\newcommand{\norm}[1]{\left\Vert #1\right\Vert}

\begin{document}
  \twocolumn[
  \begin{@twocolumnfalse}
    \noindent\centering\parbox{\textwidth}{%
      \centering {\bfseries\fontsize{14}{16}\selectfont\thetitle}\\{\fontsize{12}{14}\selectfont\theauthor}
%     \parbox{.6\linewidth}{\centering\bfseries\fontsize{14}{16}\selectfont\thetitle}\hfill%
%      \parbox{.4\linewidth}{\fontsize{12}{14}\selectfont\raggedleft\today\\\theauthor}%
  }
  
  \end{@twocolumnfalse}
  ]
  \section{Foundations of Signal Processing}
  \subsection{From Euclid to Hilbert}
  \subsubsection{Vector Spaces}
  
  \begin{definition}{(Field)} 
    Let $ F $ be a set. An \emph{operation} is a mapping $ f:F\times F \rightarrow F$. Let \emph{addition} and \emph{multiplication} be the operations $ a + b$ and $ a\cdot b, \forall a,b\in F$. Let $ c \in F $. $ F $ is a \emph{field} if it has these properties:
    \begin{enumerate}[label=(\roman*)]
      \item \emph{Associativity of addition and multiplication}: $ (a+b)+c=a+(b+c) $ and $ (a\cdot b)\cdot c=a\cdot (b\cdot c) $
      \item \emph{Commutativity of addition and multiplication}: $ a+b=b+a $ and $ a\cdot b=b\cdot a $
      \item \emph{Additive and multiplicative identity}: $ \exists 0, 1\in F: a+0=a $ and $ a\cdot 1=a $
      \item \emph{Additive and}
      
    \end{enumerate}
    
    A \emph{field} is a set F together with two operations called addition and multiplication.[1] An operation is a mapping that associates an element of the set to every pair of its elements. The result of the addition of a and b is called the sum of a and b and denoted a + b. Similarly, the result of the multiplication of a and b is called the product of a and b, and denoted ab or a⋅b. These operations are required to satisfy the following properties, referred to as field axioms. In the following definitions, a, b and c are arbitrary elements of the field F.
    
    Associativity of addition and multiplication: a + (b + c) = (a + b) + c and a · (b · c) = (a · b) · c.
    Commutativity of addition and multiplication: a + b = b + a and a · b = b · a.
    Additive and multiplicative identity: there exist two different elements 0 and 1 in F such that a + 0 = a and a · 1 = a.
    Additive inverses: for every a in F, there exists an element in F, denoted −a, called additive inverse of a, such that a + (−a) = 0.
    Multiplicative inverses: for every a ≠ 0 in F, there exists an element in F, denoted by a−1, 1/a, or 
    1
    /
    a
    , called the multiplicative inverse of a, such that a · a−1 = 1.
    Distributivity of multiplication over addition: a · (b + c) = (a · b) + (a · c) .
    
  \end{definition}

  \begin{definition}{(Vector Space)}
    A \emph{vector space} over a field of scalars $ \mathds{C} $ (or $ \mathds{R} $) is a set of vectors, $ V$, together with operations of vector addition and scalar multiplication. For any $x$, $y$, $z$ in $ V $ and $ \alpha $, $ \beta $ in $ \mathds{C} $ (or $ \mathds{R} $), these operations must satisfy the following properties:
    \begin{enumerate}[label=(\roman*)]
      \item \emph{Commutativity}: $ x + y=y+x $.
      \item \emph{Associativity}: $ (x+y)+z=x+(y+z) $ and $ (\alpha\beta)x=\alpha(\beta x) $.
      \item \emph{Distributivity}: $ \alpha(x+y)= \alpha x + \alpha y $ and $ (\alpha+\beta)x=\alpha x + \beta x .$
      \item \emph{Additive identity}: $\forall x\in V, \exists \textbf{0} \in V : x + \textbf{0} = \textbf{0} + x = x$.
      \item \emph{Additive inverse}: $ \forall x\in V, \exists -x\in V: x+(-x)=(-x)+x=\textbf{0} $
      \item \emph{Multiplicative identity}: $ \forall x\in V, 1\cdot x = x$
    \end{enumerate}
  \end{definition}
  
%  \textbf{Real plane as a vector space}\\
  Let $x\in\mathds{R}^2, x=\begin{bmatrix}x_0&x_1\end{bmatrix}^T$ be a vector. The \emph{inner product} (also \emph{scalar product} or \emph{dot product}) of $x=\begin{bmatrix}x_0&x_1\end{bmatrix}$ and $y=\begin{bmatrix}y_0&y_1\end{bmatrix}$ is $\langle x,y\rangle=x_0y_0+x_1y_1\inlineeqnum\label{e:inner_product}$. $\langle x,x\rangle=x_0^2+x_1^2$, $\langle x,x\rangle\geq 0$, $x_0=x_1=0\Rightarrow\langle x,x\rangle =0$. The \emph{norm} is $\norm{x} = \sqrt{\langle x,x\rangle}\inlineeqnum$. A \emph{unit vector} $x$ has $\norm{x}=1$. Eq \ref{e:inner_product} depends on the coordinates axes. Let $\theta_x$ the angle between the positive horizontal axis and $x$. Define $\theta_y$ similarly. So, $\langle x,y\rangle = x_0y_0+x_1y_1 = (\norm{x}\cos\theta_x)(\norm{y}\cos\theta_y)+(\norm{x}\sin\theta_x)(\norm{y}\sin\theta_y) = \norm{x}\norm{y}(\cos\theta_x\cos\theta_y+\sin\theta_x\sin\theta_y)=\norm{x}\norm{y}(\cos\theta_x\cos-\theta_y-\sin\theta_x\sin-\theta_y)=\norm{x}\norm{y}\cos(\theta_x-\theta_y) = \norm{x}\norm{y}\cos\theta \inlineeqnum$, as $\cos\theta_y=\cos-\theta_y$, $\sin(\theta_y)=-\sin(-\theta_y)$ and $\cos(a+b)=\cos a\cos b-\sin a\sin b$. 
  
  For fixed vector norms, the greater the inner product, the closer the vectors are in orientation. $\langle x,y\rangle=0$ if $\norm{x}=0$ or $ \norm{y}=0 $ (one of them is $ \begin{bmatrix}0&0 \end{bmatrix}^T $) or $ \cos\theta=0 $ ($ \theta=\pm\pi/2 $), which means they are \emph{orthogonal} or \emph{perpendicular}. The \emph{distance} is the norm of the difference: $d(x,y)=\norm{x-y}=\sqrt{\langle x-y, x-y\rangle}=\sqrt{(x-y)^2+(x-y)^2}\inlineeqnum$.
  \section{oi}
  $lkj\inlineeqnum\label{e:adf}$
  
  Eq \ref{e:adf} shows
  
	\lipsum[1-30]
\end{document}